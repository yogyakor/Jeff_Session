%  macro \factorize as above
\catcode`\_ 11

% This code (non-expandable) produces
% {{}{}{N}} followed with successive braced triplets
% {{p}{k}{m}} where p is a prime factor of N, 
% k its exponent in N, and m is the result of dividing
% N by p^k and all previous powers of smaller primes
% So the last triplet has m=1

% The code uses package xint to be able to deal
% with numbers larger than the TeX limit of 2^{31}-1
% on count registers. 

\def\factorize#1{%
    \edef\factorize_N{#1}%
    \def\factorize_exp{0}%
    \edef\factors{{{}{}{\factorize_N}}}%
    \factorize_i
}

\def\factorize_i{%
    \if\xintOdd{\factorize_N}1%
       \expandafter \factorize_ii
    \else
       \edef\factorize_exp{\xintInc{\factorize_exp}}%
       \edef\factorize_N{\xintHalf{\factorize_N}}%
       \expandafter \factorize_i
    \fi
}

\def\factorize_ii{%
    \if\xintSgn{\factorize_exp}1%
          \edef\factors{\factors{{2}{\factorize_exp}{\factorize_N}}}%
    \fi
    \if\expandafter\XINT_isOne\expandafter{\factorize_N}1%
    \else
       \def\factorize_M{3}%
       \def\factorize_exp{0}%
       \expandafter \factorize_iii
    \fi
}

\def\factorize_iii{%
    \xintAssign\xintDivision\factorize_N\factorize_M\to
        \factorize_Q\factorize_R
    \xintSgnFork{\xintSgn\factorize_R}%
        {}%
        {\edef\factorize_exp{\xintInc{\factorize_exp}}%
         \let\factorize_N\factorize_Q 
         \factorize_iii}%
        {\factorize_iv}% 
}

 \def\factorize_iv{%
    \if\xintSgn{\factorize_exp}1%
       \edef\factors{\factors{{\factorize_M}{\factorize_exp}{\factorize_N}}}%
    \fi
    \if\expandafter\XINT_isOne\expandafter{\factorize_N}1%
    \else
       % here N>1, N=QM+R (0<R<Q) is < M(Q+1) and N has no prime factors
       % at most equal to M. If a prime P>M divides N, the
       % quotient N/P will be < Q+1, hence at most Q. If Q<=M, then
       % N/P must be 1 else there would be some prime <=M dividing N.
       \if\xintGeq\factorize_M\factorize_Q 1% implies that N is prime
          \edef\factors{\factors{{\factorize_N}{1}{1}}}% we stop here
       \else % we go on testing with bigger factors
          % \edef\factorize_M{\xintInc{\xintInc{\factorize_M}}}%
          \edef\factorize_M{\xintiAdd \factorize_M 2}%
          \def\factorize_exp{0}%
          \expandafter \expandafter \expandafter \factorize_iii
       \fi
    \fi
}

\catcode`\_ 8

%% We now define the macro \FactorTree which will produce a tree
%% displaying the factorization, 
%% using TikZ+forest (for the bracket syntax, much easier to deal with compared
%% to the braces-based child-node native TikZ tree syntax)
\catcode`\_ 8

\def\auxiliarymacroa #1{\auxiliarymacrob #1}
\def\auxiliarymacrob #1#2#3{${#1}^{#2}$&\np{#3}\tabularnewline\hline}

\newcommand*\displayfactorization[1][.2\linewidth]{%
	\begin{tabular}{>{\rule{0pt}{11pt}}c|>{\raggedright}p{#1}}%
		\xintApplyUnbraced\auxiliarymacroa{\factors}
	\end{tabular}\hskip.5em plus .125em minus .125em }

%% We now define the macro \FactorTree which will produce a tree
%% displaying the factorization, 
%% using TikZ+forest (for the bracket syntax, much easier to deal with compared
%% to the braces-based child-node native TikZ tree syntax)



\makeatletter

\newtoks\FactorTreeA
\newtoks\FactorTreeB

\newcommand*{\FactorsToTree@}[3]{%
    \ifnum #2=1 % 
       \expandafter\@firstoftwo
    \else
       \expandafter\@secondoftwo
    \fi
    % exponent #2 is 1, so don't print it
    {\xintSgnFork{\xintCmp{#3}{1}}% check to see if end has been reached
    {}%
        % here, this is the last triplet and it has the shape {P}{1}{1}
        % and P was already inserted as tree node in the previous step
        % and the forest syntax allows to insert options here
    {\FactorTreeA\expandafter{\the\FactorTreeA ,draw,circle}}%
    {\FactorTreeA\expandafter{\the\FactorTreeA 
                             [{${#1}$}]
                             [{${#3}$}%
                             }%
     \FactorTreeB\expandafter{\the\FactorTreeB ]}%
    }}%
    % exponent #2 is > 1
    {\xintSgnFork{\xintCmp{#3}{1}}% check to see if end has been reached
    {}%
    {\FactorTreeA\expandafter{\the\FactorTreeA
                              [{${#1}^{#2}$}]
                              %[1]%
                              }%
    }%
    {\FactorTreeA\expandafter{\the\FactorTreeA 
                             [{${#1}^{#2}$}]
                             [{$#3$}%
                             }%
     \FactorTreeB\expandafter{\the\FactorTreeB ]}%
    }}%
}

\newcommand*{\FactorsToTree}[1]{\FactorsToTree@ #1}

% for factors displayed inline 

\def\@factorinliner  #1{\@factorinliner@ #1}
\def\@factorinliner@ #1#2#3{\ifnum #2>1 \expandafter\@firstoftwo\else
                                        \expandafter\@secondoftwo\fi
                            {{#1}^{#2}}{#1}}
\def\FactorsInline{%
    \xintListWithSep\times
             {\xintApply\@factorinliner{\expandafter\@gobble\factors}}%
}

\newlength{\horizontalshift}  % for positioning of the edges from the tree top
\newsavebox{\NandFactors}

\newcommand*{\FactorTree}[1]{%
    \factorize{#1}%
    \sbox{\NandFactors}{$#1=\FactorsInline$}%
    \setlength{\horizontalshift}{(\wd\NandFactors-\widthof{$#1$})/2}%
    \FactorTreeA{}%
    \FactorTreeB{}%
    \bracketset{action character=@}%
    \xintApplyUnbraced\FactorsToTree{\expandafter\@gobble\factors}%
    \begin{forest}
      for tree={edge path={\noexpand\path [\forestoption{edge}]
                                (!u.south)--(.child anchor);}},
      where={level()==1}
        {edge path= {\noexpand\path [\forestoption{edge}]
            ($(!u.south)-(\the\horizontalshift,0cm)$)--(.child anchor);}}{},
      where={n()==1}{draw,circle}{},
      [{\box\NandFactors}, right=\horizontalshift,
      @\the\FactorTreeA
      @\the\FactorTreeB
      ]
    \end{forest}
}

\makeatother