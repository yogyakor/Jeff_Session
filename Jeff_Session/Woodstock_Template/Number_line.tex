\makeatletter
\newif\ifnl@mixednumbers
\newif\ifnl@ticksabove
\newif\ifnl@isnumberto
\tikzset{% http://tex.stackexchange.com/a/159856/ - Claudio Fiandrino
  number line/.code={
    \tikzset{
      /number line/.cd,%
      #1
    }
  },
  /number line/.cd,
  dot/.store in=\nl@dot,
  dot opacity/.store in=\nl@dotopacity,
  fill/.store in=\nl@fill,
  fraction/.store in=\nl@fraction,
  h scale/.store in=\nl@hscale,
  line height/.store in=\nl@lineht,
  line above/.store in=\nl@lineabove,
  line below/.store in=\nl@linebelow,
  mark at/.store in=\nl@markat,
  max/.store in=\nl@max,
  min/.store in=\nl@min,
  mixed numbers/.is if=nl@mixednumbers,
  mixed numbers/.default=true,
  number to/.store in=\nl@numberto,
  number from/.store in=\nl@numberfrom,
  ticks above/.is if=nl@ticksabove,
  ticks above/.default=true,
  v scale/.store in=\nl@vscale,
  fraction=4,
  v scale=1.2,
  h scale=4,
  max=3,
  min=0,
  number to={7/4},
  number from=0,
  mixed numbers=false,
  fill=orange,
  dot=green,
  dot opacity=.75,
  ticks above,
  mark at=,
  line height=0.2,
  line above=0.2,
  line below=0.2,
}
\newcommand*\tnl{% modified from ref: WeCanLearnAnything at http://tex.stackexchange.com/questions/267921/macro-for-mixed-numbers-on-number-line-tikz (but I doubt this is the original source)
  \begin{scope}[xscale=\nl@hscale,yscale=\nl@vscale]
    \edef\tempa{none}%
    \edef\tempb{\nl@numberto}%
    \ifx\tempa\tempb
      \nl@isnumbertofalse
    \else\nl@isnumbertotrue
      \filldraw[\nl@fill] (\nl@numberfrom,0) rectangle (\nl@numberto,\nl@lineht);% shaded portion of number line
    \fi
    \draw
    (\nl@min,0)--(\nl@max,0)% lower part of x-axis
    (\nl@min,\nl@lineht)--(\nl@max,\nl@lineht);% higher part of x-axis
    \foreach \x in {\nl@min,...,\nl@max}
      \node [anchor=mid] at (\x,-0.5) {\x};% whole numbers underneath number line
    \pgfmathsetmacro\nl@maxfrac{int(\nl@max*\nl@fraction)}
    \pgfmathsetmacro\nl@minfrac{int(\nl@min*\nl@fraction)}
    \foreach \x in {\nl@minfrac,...,\nl@maxfrac}% fractional tick marks and numbers above number line
    {%
      \draw (\x/\nl@fraction,-\nl@linebelow)--(\x/\nl@fraction,\nl@lineht);
      \ifnl@ticksabove
        \node[above] at (\x/\nl@fraction,{\nl@lineht+0.05}) {$\frac{\x}{\nl@fraction}$}; % draw tick and mark fraction above line
      \fi
      \ifnl@mixednumbers
        \pgfmathsetmacro\intbit{int(\x/\nl@fraction)}% integer bit
        \pgfmathsetmacro\fracbit{int(\x-\nl@fraction*\intbit)}% proper fraction bit
        \ifnum\intbit=0\let\intbit\relax\fi % don't print 0 in mixed numbers
        \ifnum\fracbit=0\else
          \node [anchor=mid] at (\x/\nl@fraction,{-\nl@linebelow-0.3}) {$\intbit\frac{\fracbit}{\nl@fraction}$}; % mark the mixed number below the line
        \fi
      \fi
    }%
    \ifnl@isnumberto
      \fill[\nl@dot,opacity=\nl@dotopacity] (\nl@numberto,.5*\nl@lineht) circle[x radius=\nl@lineht cm/\nl@hscale,y radius=\nl@lineht cm/\nl@vscale];% dot
    \fi
    \foreach \i in \nl@markat
    {%
      \draw (\i,0) -- (\i,{\nl@lineht+\nl@lineabove}) node [above] {\i};% draw tick to node above line to mark point specified with mark at
      \fill[\nl@dot,opacity=\nl@dotopacity] (\i,.5*\nl@lineht) circle[x radius=\nl@lineht cm/\nl@hscale,y radius=\nl@lineht cm/\nl@vscale];% dot
    };
  \end{scope}}
\NewDocumentCommand \NumberLine { s O {} }{%
  \IfBooleanTF {#1}{%
    \tikz[number line={mixed numbers=false,#2}]\tnl;%
  }{%
    \tikz[number line={mixed numbers=true,#2}]\tnl;%
  }}
\makeatother



%\NumberLine*[
%  dot=blue,
%  dot opacity=.75,
%  fill=blue,
%  fraction=1,
%  ticks above=false,
%  min=-3,
%  number to=none,
%  h scale=5,
%  mark at={13/6,14/6,0.7,0.9},
%]

