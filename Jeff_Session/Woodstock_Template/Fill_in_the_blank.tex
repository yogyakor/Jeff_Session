%\documentclass[fleqn]{article}
%\documentclass{beamer}

%\usepackage{xparse}
%\usepackage{tikz}
%   \usetikzlibrary{calc}
%\usepackage{mathtools}

\makeatletter
\newlength\fillb@width
\def\fillb@widthfactor{1.75}
\newif\iffillbhideanswer
\fillbhideanswertrue
\tikzset{
   every fill in box/.style={
     inner xsep=0pt,
     minimum height=3ex,
     align=center,
     font={\sffamily\slshape},
   },
   colored box/.style={
      every fill in box,
     % fill=yellow!50!white,
       fill=orange!28!white,
   },
   framed box/.style={
      every fill in box,
      draw,
   },
   underline style/.style={},
   underlined box/.style={
      every fill in box,
      append after command={%
         \pgfextra{\begin{pgfinterruptpath}
            \draw [underline style] (\tikzlastnode.south west)
                  -- (\tikzlastnode.south east);
         \end{pgfinterruptpath}}
      },
   },
   bracket style/.style={},
   underbracked box/.style={
      every fill in box,
      append after command={%
         \pgfextra{\begin{pgfinterruptpath}
            \draw [bracket style] ($(\tikzlastnode.south west)+(0,2pt)$)
                  |- (\tikzlastnode.south)
                  -| ($(\tikzlastnode.south east)+(0,2pt)$);
         \end{pgfinterruptpath}}
      },
   },
   underoverbracked box/.style={
      every fill in box,
      append after command={%
         \pgfextra{\begin{pgfinterruptpath}
            \draw [bracket style] ($(\tikzlastnode.north west)-(0,2pt)$)
                  |- (\tikzlastnode.north)
                  -| ($(\tikzlastnode.north east)-(0,2pt)$);
            \draw [bracket style] ($(\tikzlastnode.south west)+(0,2pt)$)
                  |- (\tikzlastnode.south)
                  -| ($(\tikzlastnode.south east)+(0,2pt)$);
         \end{pgfinterruptpath}}
      },
   },
   fill in/.style={
      colored box,
      framed box,
   },
}
\NewDocumentCommand { \fillb@hide } { m } {%
   \iffillbhideanswer
      \phantom{#1}%
   \else
      #1%
   \fi
}
\NewDocumentCommand { \fillb@makebox }{ m }{%
   \settowidth{\fillb@width}{\tikz\node[fill in]{#1};}%
   \begin{tikzpicture}[baseline=(fill in node.base)]
      \node (fill in node) [text width=\fillb@widthfactor*\fillb@width,fill in] {%
         \fillb@hide{#1}%
      };
   \end{tikzpicture}%
}
\NewDocumentCommand { \fillb } { s d{<}{>} o m }{{%
   \IfBooleanT{#1}{\fillbhideanswerfalse}%
   \IfValueT{#2}{\only<#2>{\fillbhideanswerfalse}}%
   \IfValueT{#3}{\tikzset{fill in/.style={#3}}}%
   \ifmmode
      \mathchoice
         {\fillb@makebox{$\displaystyle#4$}}
         {\fillb@makebox{$\textstyle#4$}}
         {\fillb@makebox{$\scriptstyle#4$}}
         {\fillb@makebox{$\scriptscriptstyle#4$}}
   \else
      \fillb@makebox{#4}%
   \fi
   \IfValueT{#2}{}%
}}
\makeatother

%\begin{document}
% STYLE SETTING EXAMPLES
%\tikzset{colored box/.append style={fill=black!15}}
%\tikzset{fill in/.style={framed box}}
%\tikzset{fill in/.style={underlined box}}
%\tikzset{underline style/.style={densely dotted,thick}}
%\tikzset{fill in/.style={underbracked box}}
%\tikzset{fill in/.style={underoverbracked box}}
%\tikzset{bracket style/.style={gray,thick}}
%\fillbhideanswerfalse


% ARTICLE/BOOK EXAMPLES

%A \fib{short} example with math $1 + 2^{\fib{2}} = \fib{5} = \sqrt{25}$.

%\vspace{2cm}
%In \fib{text} mode and math $1 + 3 = \fib{4} = \fib{\frac{8}{2}}$
%\begin{equation}
%1 + 3 = \fib{4} = \fib{\frac{8}{2}}
%\end{equation}
%\begin{equation}
%(a + b)^2 = \fib{a^2 + 2ab + b^2}
%\end{equation}
%\begin{equation}
%\begin{pmatrix}
%1 \\ 2 \\ 3
%\end{pmatrix}
%\times
%\begin{pmatrix}
%4 \\ 5 \\ 6
%\end{pmatrix}
%=
%\fib{\begin{pmatrix}
%	-3 \\ 6 \\ -3
%	\end{pmatrix}}
%\end{equation}
%With an asterisk, i.e. \verb+\fib*+, the \fib*{solution} is always visible.
%The optional argument can be used to change \fib*[underlined box]{styles} locally.


% BEAMER EXAMPLES
%\begin{frame}{Beamer example 1}
%   \only<2->{\fillbhideanswerfalse}
%   A \fillb{short} example with math $1 + 2^{\fillb{2}} = \fillb{5} =
%   \sqrt{25}$.
%\end{frame}
%\begin{frame}{Beamer example 2}
%   A \fillb<2->{short} example with math $1 + 2^{\fillb<3->{2}} =
%   \fillb<4->{5} = \sqrt{25}$.
%\end{frame}

%\end{document}