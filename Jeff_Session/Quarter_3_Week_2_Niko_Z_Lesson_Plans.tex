\documentclass[12pt,compress,leqno]{beamer}
%\includeonlyframes{plan}
\setbeamertemplate{blocks}[rounded][shadow=true]
\setbeamertemplate{navigation symbols}{}%remove navigation symbols
%\setbeamercovered{transparent}
\setbeamercolor{alerted text}{fg=magenta}

%\usetheme{Ilmenau}
\usetheme{Warsaw} %goes with shadow outer theme
%\usetheme{AnnArbor}
%\usetheme{Malmoe}
%\usetheme{Goettingen}
%\usetheme{JLTree}
%\usetheme{metropolis}
%\usetheme{progressbar}
    %\progressbaroptions{frametitle=picture-section,
    %headline=sections,imagename=images/woodstock2}
    % headline=none or headline=sections

%\useoutertheme{infolines}
%\useoutertheme{split}  %split is included in Malmoe beamer theme 

%\usecolortheme{owl}
%\usecolortheme{beaver} %dark red text, gray header, white background
%\usecolortheme{crane}
%\usecolortheme{spruce}
%\usecolortheme{seahorse}
%\usecolortheme{monarca}
%\usecolortheme{albatross}
%\usecolortheme{whale}
%\usecolortheme{dolphin}
%\usecolortheme{magpie}
\usecolortheme[named=purple]{structure} % purple, cyan, brown, red

%\usefonttheme{professionalfonts}
\usepackage{iwona}
%\usepackage{FiraSans}
%\usepackage{FiraMono}

%\usepackage[latin1]{inputenc}
\usepackage[T1]{fontenc}
\usepackage{multicol}
\usepackage{amsmath}
\usepackage{amsfonts}
\usepackage{wasysym}
\usepackage{gensymb}
\usepackage{amssymb}
\usepackage[misc]{ifsym} %for dice
\usepackage{epsdice}
\usepackage{graphicx}
\usepackage{xcolor}
\usepackage{wrapfig}
%\usepackage{IEEEtrantools}

\usepackage{calc} % for \widthof
\usepackage{tikz}
\usetikzlibrary{calc,arrows}
\usepackage{forest}
\usepackage{xint}
\usepackage{xintexpr}
\usepackage{array}
\usepackage[np]{numprint}
\usepackage{xparse}
\usepackage{mathtools}

\usepackage{pstricks-add}
\usetikzlibrary{matrix}
\usepackage{tkz-euclide}
\usetkzobj[all]
\usepackage{xlop}
\usepackage{tcolorbox}
\usepackage{pgfplots}
\usepackage{pgfplotstable} % For \pgfplotstableread
\pgfplotsset{width=1\linewidth,compat=1.9}
\usepackage{pgf-pie}
\usepackage[makeroom]{cancel}
\usepackage{xmpmulti}
\usepackage{xparse}
\usepackage{scalerel}
%\usepackage{prerex}

%This section loads external code for specific math related tasks
%  macro \factorize as above
\catcode`\_ 11

% This code (non-expandable) produces
% {{}{}{N}} followed with successive braced triplets
% {{p}{k}{m}} where p is a prime factor of N, 
% k its exponent in N, and m is the result of dividing
% N by p^k and all previous powers of smaller primes
% So the last triplet has m=1

% The code uses package xint to be able to deal
% with numbers larger than the TeX limit of 2^{31}-1
% on count registers. 

\def\factorize#1{%
    \edef\factorize_N{#1}%
    \def\factorize_exp{0}%
    \edef\factors{{{}{}{\factorize_N}}}%
    \factorize_i
}

\def\factorize_i{%
    \if\xintOdd{\factorize_N}1%
       \expandafter \factorize_ii
    \else
       \edef\factorize_exp{\xintInc{\factorize_exp}}%
       \edef\factorize_N{\xintHalf{\factorize_N}}%
       \expandafter \factorize_i
    \fi
}

\def\factorize_ii{%
    \if\xintSgn{\factorize_exp}1%
          \edef\factors{\factors{{2}{\factorize_exp}{\factorize_N}}}%
    \fi
    \if\expandafter\XINT_isOne\expandafter{\factorize_N}1%
    \else
       \def\factorize_M{3}%
       \def\factorize_exp{0}%
       \expandafter \factorize_iii
    \fi
}

\def\factorize_iii{%
    \xintAssign\xintDivision\factorize_N\factorize_M\to
        \factorize_Q\factorize_R
    \xintSgnFork{\xintSgn\factorize_R}%
        {}%
        {\edef\factorize_exp{\xintInc{\factorize_exp}}%
         \let\factorize_N\factorize_Q 
         \factorize_iii}%
        {\factorize_iv}% 
}

 \def\factorize_iv{%
    \if\xintSgn{\factorize_exp}1%
       \edef\factors{\factors{{\factorize_M}{\factorize_exp}{\factorize_N}}}%
    \fi
    \if\expandafter\XINT_isOne\expandafter{\factorize_N}1%
    \else
       % here N>1, N=QM+R (0<R<Q) is < M(Q+1) and N has no prime factors
       % at most equal to M. If a prime P>M divides N, the
       % quotient N/P will be < Q+1, hence at most Q. If Q<=M, then
       % N/P must be 1 else there would be some prime <=M dividing N.
       \if\xintGeq\factorize_M\factorize_Q 1% implies that N is prime
          \edef\factors{\factors{{\factorize_N}{1}{1}}}% we stop here
       \else % we go on testing with bigger factors
          % \edef\factorize_M{\xintInc{\xintInc{\factorize_M}}}%
          \edef\factorize_M{\xintiAdd \factorize_M 2}%
          \def\factorize_exp{0}%
          \expandafter \expandafter \expandafter \factorize_iii
       \fi
    \fi
}

\catcode`\_ 8

%% We now define the macro \FactorTree which will produce a tree
%% displaying the factorization, 
%% using TikZ+forest (for the bracket syntax, much easier to deal with compared
%% to the braces-based child-node native TikZ tree syntax)
\catcode`\_ 8

\def\auxiliarymacroa #1{\auxiliarymacrob #1}
\def\auxiliarymacrob #1#2#3{${#1}^{#2}$&\np{#3}\tabularnewline\hline}

\newcommand*\displayfactorization[1][.2\linewidth]{%
	\begin{tabular}{>{\rule{0pt}{11pt}}c|>{\raggedright}p{#1}}%
		\xintApplyUnbraced\auxiliarymacroa{\factors}
	\end{tabular}\hskip.5em plus .125em minus .125em }

%% We now define the macro \FactorTree which will produce a tree
%% displaying the factorization, 
%% using TikZ+forest (for the bracket syntax, much easier to deal with compared
%% to the braces-based child-node native TikZ tree syntax)



\makeatletter

\newtoks\FactorTreeA
\newtoks\FactorTreeB

\newcommand*{\FactorsToTree@}[3]{%
    \ifnum #2=1 % 
       \expandafter\@firstoftwo
    \else
       \expandafter\@secondoftwo
    \fi
    % exponent #2 is 1, so don't print it
    {\xintSgnFork{\xintCmp{#3}{1}}% check to see if end has been reached
    {}%
        % here, this is the last triplet and it has the shape {P}{1}{1}
        % and P was already inserted as tree node in the previous step
        % and the forest syntax allows to insert options here
    {\FactorTreeA\expandafter{\the\FactorTreeA ,draw,circle}}%
    {\FactorTreeA\expandafter{\the\FactorTreeA 
                             [{${#1}$}]
                             [{${#3}$}%
                             }%
     \FactorTreeB\expandafter{\the\FactorTreeB ]}%
    }}%
    % exponent #2 is > 1
    {\xintSgnFork{\xintCmp{#3}{1}}% check to see if end has been reached
    {}%
    {\FactorTreeA\expandafter{\the\FactorTreeA
                              [{${#1}^{#2}$}]
                              %[1]%
                              }%
    }%
    {\FactorTreeA\expandafter{\the\FactorTreeA 
                             [{${#1}^{#2}$}]
                             [{$#3$}%
                             }%
     \FactorTreeB\expandafter{\the\FactorTreeB ]}%
    }}%
}

\newcommand*{\FactorsToTree}[1]{\FactorsToTree@ #1}

% for factors displayed inline 

\def\@factorinliner  #1{\@factorinliner@ #1}
\def\@factorinliner@ #1#2#3{\ifnum #2>1 \expandafter\@firstoftwo\else
                                        \expandafter\@secondoftwo\fi
                            {{#1}^{#2}}{#1}}
\def\FactorsInline{%
    \xintListWithSep\times
             {\xintApply\@factorinliner{\expandafter\@gobble\factors}}%
}

\newlength{\horizontalshift}  % for positioning of the edges from the tree top
\newsavebox{\NandFactors}

\newcommand*{\FactorTree}[1]{%
    \factorize{#1}%
    \sbox{\NandFactors}{$#1=\FactorsInline$}%
    \setlength{\horizontalshift}{(\wd\NandFactors-\widthof{$#1$})/2}%
    \FactorTreeA{}%
    \FactorTreeB{}%
    \bracketset{action character=@}%
    \xintApplyUnbraced\FactorsToTree{\expandafter\@gobble\factors}%
    \begin{forest}
      for tree={edge path={\noexpand\path [\forestoption{edge}]
                                (!u.south)--(.child anchor);}},
      where={level()==1}
        {edge path= {\noexpand\path [\forestoption{edge}]
            ($(!u.south)-(\the\horizontalshift,0cm)$)--(.child anchor);}}{},
      where={n()==1}{draw,circle}{},
      [{\box\NandFactors}, right=\horizontalshift,
      @\the\FactorTreeA
      @\the\FactorTreeB
      ]
    \end{forest}
}

\makeatother
%\documentclass[fleqn]{article}
%\documentclass{beamer}

%\usepackage{xparse}
%\usepackage{tikz}
%   \usetikzlibrary{calc}
%\usepackage{mathtools}

\makeatletter
\newlength\fillb@width
\def\fillb@widthfactor{1.75}
\newif\iffillbhideanswer
\fillbhideanswertrue
\tikzset{
   every fill in box/.style={
     inner xsep=0pt,
     minimum height=3ex,
     align=center,
     font={\sffamily\slshape},
   },
   colored box/.style={
      every fill in box,
     % fill=yellow!50!white,
       fill=orange!28!white,
   },
   framed box/.style={
      every fill in box,
      draw,
   },
   underline style/.style={},
   underlined box/.style={
      every fill in box,
      append after command={%
         \pgfextra{\begin{pgfinterruptpath}
            \draw [underline style] (\tikzlastnode.south west)
                  -- (\tikzlastnode.south east);
         \end{pgfinterruptpath}}
      },
   },
   bracket style/.style={},
   underbracked box/.style={
      every fill in box,
      append after command={%
         \pgfextra{\begin{pgfinterruptpath}
            \draw [bracket style] ($(\tikzlastnode.south west)+(0,2pt)$)
                  |- (\tikzlastnode.south)
                  -| ($(\tikzlastnode.south east)+(0,2pt)$);
         \end{pgfinterruptpath}}
      },
   },
   underoverbracked box/.style={
      every fill in box,
      append after command={%
         \pgfextra{\begin{pgfinterruptpath}
            \draw [bracket style] ($(\tikzlastnode.north west)-(0,2pt)$)
                  |- (\tikzlastnode.north)
                  -| ($(\tikzlastnode.north east)-(0,2pt)$);
            \draw [bracket style] ($(\tikzlastnode.south west)+(0,2pt)$)
                  |- (\tikzlastnode.south)
                  -| ($(\tikzlastnode.south east)+(0,2pt)$);
         \end{pgfinterruptpath}}
      },
   },
   fill in/.style={
      colored box,
      framed box,
   },
}
\NewDocumentCommand { \fillb@hide } { m } {%
   \iffillbhideanswer
      \phantom{#1}%
   \else
      #1%
   \fi
}
\NewDocumentCommand { \fillb@makebox }{ m }{%
   \settowidth{\fillb@width}{\tikz\node[fill in]{#1};}%
   \begin{tikzpicture}[baseline=(fill in node.base)]
      \node (fill in node) [text width=\fillb@widthfactor*\fillb@width,fill in] {%
         \fillb@hide{#1}%
      };
   \end{tikzpicture}%
}
\NewDocumentCommand { \fillb } { s d{<}{>} o m }{{%
   \IfBooleanT{#1}{\fillbhideanswerfalse}%
   \IfValueT{#2}{\only<#2>{\fillbhideanswerfalse}}%
   \IfValueT{#3}{\tikzset{fill in/.style={#3}}}%
   \ifmmode
      \mathchoice
         {\fillb@makebox{$\displaystyle#4$}}
         {\fillb@makebox{$\textstyle#4$}}
         {\fillb@makebox{$\scriptstyle#4$}}
         {\fillb@makebox{$\scriptscriptstyle#4$}}
   \else
      \fillb@makebox{#4}%
   \fi
   \IfValueT{#2}{}%
}}
\makeatother

%\begin{document}
% STYLE SETTING EXAMPLES
%\tikzset{colored box/.append style={fill=black!15}}
%\tikzset{fill in/.style={framed box}}
%\tikzset{fill in/.style={underlined box}}
%\tikzset{underline style/.style={densely dotted,thick}}
%\tikzset{fill in/.style={underbracked box}}
%\tikzset{fill in/.style={underoverbracked box}}
%\tikzset{bracket style/.style={gray,thick}}
%\fillbhideanswerfalse


% ARTICLE/BOOK EXAMPLES

%A \fib{short} example with math $1 + 2^{\fib{2}} = \fib{5} = \sqrt{25}$.

%\vspace{2cm}
%In \fib{text} mode and math $1 + 3 = \fib{4} = \fib{\frac{8}{2}}$
%\begin{equation}
%1 + 3 = \fib{4} = \fib{\frac{8}{2}}
%\end{equation}
%\begin{equation}
%(a + b)^2 = \fib{a^2 + 2ab + b^2}
%\end{equation}
%\begin{equation}
%\begin{pmatrix}
%1 \\ 2 \\ 3
%\end{pmatrix}
%\times
%\begin{pmatrix}
%4 \\ 5 \\ 6
%\end{pmatrix}
%=
%\fib{\begin{pmatrix}
%	-3 \\ 6 \\ -3
%	\end{pmatrix}}
%\end{equation}
%With an asterisk, i.e. \verb+\fib*+, the \fib*{solution} is always visible.
%The optional argument can be used to change \fib*[underlined box]{styles} locally.


% BEAMER EXAMPLES
%\begin{frame}{Beamer example 1}
%   \only<2->{\fillbhideanswerfalse}
%   A \fillb{short} example with math $1 + 2^{\fillb{2}} = \fillb{5} =
%   \sqrt{25}$.
%\end{frame}
%\begin{frame}{Beamer example 2}
%   A \fillb<2->{short} example with math $1 + 2^{\fillb<3->{2}} =
%   \fillb<4->{5} = \sqrt{25}$.
%\end{frame}

%\end{document}
\input{longdiv}
\makeatletter
\newif\ifnl@mixednumbers
\newif\ifnl@ticksabove
\newif\ifnl@isnumberto
\tikzset{% http://tex.stackexchange.com/a/159856/ - Claudio Fiandrino
  number line/.code={
    \tikzset{
      /number line/.cd,%
      #1
    }
  },
  /number line/.cd,
  dot/.store in=\nl@dot,
  dot opacity/.store in=\nl@dotopacity,
  fill/.store in=\nl@fill,
  fraction/.store in=\nl@fraction,
  h scale/.store in=\nl@hscale,
  line height/.store in=\nl@lineht,
  line above/.store in=\nl@lineabove,
  line below/.store in=\nl@linebelow,
  mark at/.store in=\nl@markat,
  max/.store in=\nl@max,
  min/.store in=\nl@min,
  mixed numbers/.is if=nl@mixednumbers,
  mixed numbers/.default=true,
  number to/.store in=\nl@numberto,
  number from/.store in=\nl@numberfrom,
  ticks above/.is if=nl@ticksabove,
  ticks above/.default=true,
  v scale/.store in=\nl@vscale,
  fraction=4,
  v scale=1.2,
  h scale=4,
  max=3,
  min=0,
  number to={7/4},
  number from=0,
  mixed numbers=false,
  fill=orange,
  dot=green,
  dot opacity=.75,
  ticks above,
  mark at=,
  line height=0.2,
  line above=0.2,
  line below=0.2,
}
\newcommand*\tnl{% modified from ref: WeCanLearnAnything at http://tex.stackexchange.com/questions/267921/macro-for-mixed-numbers-on-number-line-tikz (but I doubt this is the original source)
  \begin{scope}[xscale=\nl@hscale,yscale=\nl@vscale]
    \edef\tempa{none}%
    \edef\tempb{\nl@numberto}%
    \ifx\tempa\tempb
      \nl@isnumbertofalse
    \else\nl@isnumbertotrue
      \filldraw[\nl@fill] (\nl@numberfrom,0) rectangle (\nl@numberto,\nl@lineht);% shaded portion of number line
    \fi
    \draw
    (\nl@min,0)--(\nl@max,0)% lower part of x-axis
    (\nl@min,\nl@lineht)--(\nl@max,\nl@lineht);% higher part of x-axis
    \foreach \x in {\nl@min,...,\nl@max}
      \node [anchor=mid] at (\x,-0.5) {\x};% whole numbers underneath number line
    \pgfmathsetmacro\nl@maxfrac{int(\nl@max*\nl@fraction)}
    \pgfmathsetmacro\nl@minfrac{int(\nl@min*\nl@fraction)}
    \foreach \x in {\nl@minfrac,...,\nl@maxfrac}% fractional tick marks and numbers above number line
    {%
      \draw (\x/\nl@fraction,-\nl@linebelow)--(\x/\nl@fraction,\nl@lineht);
      \ifnl@ticksabove
        \node[above] at (\x/\nl@fraction,{\nl@lineht+0.05}) {$\frac{\x}{\nl@fraction}$}; % draw tick and mark fraction above line
      \fi
      \ifnl@mixednumbers
        \pgfmathsetmacro\intbit{int(\x/\nl@fraction)}% integer bit
        \pgfmathsetmacro\fracbit{int(\x-\nl@fraction*\intbit)}% proper fraction bit
        \ifnum\intbit=0\let\intbit\relax\fi % don't print 0 in mixed numbers
        \ifnum\fracbit=0\else
          \node [anchor=mid] at (\x/\nl@fraction,{-\nl@linebelow-0.3}) {$\intbit\frac{\fracbit}{\nl@fraction}$}; % mark the mixed number below the line
        \fi
      \fi
    }%
    \ifnl@isnumberto
      \fill[\nl@dot,opacity=\nl@dotopacity] (\nl@numberto,.5*\nl@lineht) circle[x radius=\nl@lineht cm/\nl@hscale,y radius=\nl@lineht cm/\nl@vscale];% dot
    \fi
    \foreach \i in \nl@markat
    {%
      \draw (\i,0) -- (\i,{\nl@lineht+\nl@lineabove}) node [above] {\i};% draw tick to node above line to mark point specified with mark at
      \fill[\nl@dot,opacity=\nl@dotopacity] (\i,.5*\nl@lineht) circle[x radius=\nl@lineht cm/\nl@hscale,y radius=\nl@lineht cm/\nl@vscale];% dot
    };
  \end{scope}}
\NewDocumentCommand \NumberLine { s O {} }{%
  \IfBooleanTF {#1}{%
    \tikz[number line={mixed numbers=false,#2}]\tnl;%
  }{%
    \tikz[number line={mixed numbers=true,#2}]\tnl;%
  }}
\makeatother



%\NumberLine*[
%  dot=blue,
%  dot opacity=.75,
%  fill=blue,
%  fraction=1,
%  ticks above=false,
%  min=-3,
%  number to=none,
%  h scale=5,
%  mark at={13/6,14/6,0.7,0.9},
%]


\makeatletter
\def\tikz@collect@child hild{% 
	\pgfutil@ifnextchar<{\tikz@collect@child@overlay}% 
	{\pgfutil@ifnextchar[{\tikz@collect@childA}{\tikz@collect@childA[]}}% 
} 
\def\tikz@collect@child@overlay<#1> 

\def\tikz@collect@child@@overlay#1[{\tikz@collect@childA[child overlay={#1},} 

\def\tikzprocessoverlay#1#2#3{% 
	\def\beamer@doifinframe{#2}% 
	\def\beamer@doifnotinframe{#3}% 
	\beamer@masterdecode{#1}% 
	\beamer@donow% 
} 

% Extra hackery to allow preactions on different layers. 
% 
\def\tikz@extra@preaction#1{% 
	{% 
		\pgfsys@beginscope% 
		\setbox\tikz@figbox=\box\voidb@x% 
		\begingroup\tikzset{#1}\expandafter\endgroup% 
		\expandafter\def\expandafter\tikz@preaction@layer\expandafter{\tikz@preaction@layer}% 
		\ifx\tikz@preaction@layer\pgfutil@empty% 
		\path[#1];% do extra path 
		\else% 
		\begin{pgfonlayer}{\tikz@preaction@layer}% 
			\path[#1];% 
		\end{pgfonlayer} 
		\fi% 
		\pgfsyssoftpath@setcurrentpath\tikz@actions@path% restore 
		\tikz@restorepathsize% 
		\pgfsys@endscope% 
	}% 
} 
\let\tikz@preaction@layer=\pgfutil@empty 

\tikzset{preaction layer/.store in=\tikz@preaction@layer} 

\makeatother

\tikzset{% 
	child overlay/.code={% 
		\tikzprocessoverlay{#1}{}% 
		{% 
			\tikzset{% 
				edge from parent/.style={draw=none}, 
				every node/.style={ 
					draw=none, fill=none, 
					execute at begin node={\setbox0=\hbox\bgroup\hskip0pt\let\\=\relax}, 
					execute at end node=\egroup\phantom{\box0} 
				},%
				bag/.style={draw=none}%
			}% 
		}% 
	} 
}


%This section defines custom color commands

\definecolor{airforceblue}{rgb}{0.36, 0.54, 0.66}
\definecolor{crimson}{rgb}{0.86, 0.08, 0.24}

\definecolor{kugreen}{RGB}{50,93,61}
\definecolor{kugreenlys}{RGB}{132,158,139}
\definecolor{kugreenlyslys}{RGB}{173,190,177}
\definecolor{kugreenlyslyslys}{RGB}{214,223,216}

%\usecolortheme[named=kugreen]{structure}
%\usecolortheme[named=crimson]{structure}

%\newcommand{\blue}[1]{\textcolor{blue}{#1}}
\newcommand{\bluebf}[1]{\textbf{\textcolor{blue}{#1}}}
%\newcommand{\red}[1]{\textcolor{red}{#1}}
\newcommand{\redbf}[1]{\textbf{\textcolor{red}{#1}}}
\newcommand{\purple}[1]{\textcolor{purple}{#1}}
\newcommand{\purplebf}[1]{\textbf{\textcolor{purple}{#1}}}
\newcommand{\greenbf}[1]{\textbf{\textcolor{green!70!black}{#1}}}
\newcommand{\kugreenbf}[1]{\textbf{\textcolor{kugreen!70!black}{#1}}}
\newcommand{\brown}[1]{\textcolor{brown}{#1}}
\newcommand{\brownbf}[1]{\textbf{\textcolor{brown}{#1}}}
\newcommand{\crimson}[1]{\textcolor{crimson}{#1}}
\newcommand{\crimsonbf}[1]{\textbf{\textcolor{crimson}{#1}}}
\newcommand{\airforceblue}[1]{\textcolor{airforceblue}{#1}}
\newcommand{\airforcebluebf}[1]{\textbf{\textcolor{airforceblue}{#1}}}

%\newcommand{\green}[1]{\textcolor{green}{#1}}
\newcommand{\kugreen}[1]{\textcolor{kugreen}{#1}}
\newcommand{\orange}[1]{\textcolor{orange}{#1}}
\newcommand{\orangebf}[1]{\textbf{\textcolor{orange}{#1}}}


%This section defines custom symbols and signs commonly used throughout presentation

\newcommand{\mangahightext}{\textbf{{\textcolor{orange!90!black}{Manga}\textcolor{airforceblue}{High }}}\includegraphics[width=0.04\linewidth]{Images/mangahighlogo}\hspace{3pt}}

\newcommand{\mangahighlogo}{\includegraphics[width=0.15\linewidth]{Images/mangahigh}\hspace{3pt}}
\newcommand{\mangahighheader}{\includegraphics[width=0.24\linewidth]{Images/mangahighheader}\hspace{5pt}}
\newcommand{\khanacademy}{\includegraphics[width=0.03\linewidth]{Images/khanacademyleaf}{\hspace{1pt}\bf KHAN}ACADEMY\hspace{3pt}}
\newcommand{\mathster}{\textbf{\textcolor{red}{Mathster }}}
\newcommand{\webassign}{\textbf{\textit{Web}}\textbf{\textcolor{crimson!50!red}{Assign }}}
\newcommand{\textbook}{\redbf{Textbook }}
\newcommand{\Textbook}{\redbf{\TeX tbook }}
\newcommand{\Texbook}{\redbf{\TeX tbook }}
\newcommand{\whiteboards}{\greenbf{Whiteboards }}
\newcommand{\saxontwo}{\href{http://saxontwo.nikoz.academy}{\textit{\airforcebluebf{http://saxontwo.nikoz.academy}}}}
\newcommand{\saxonthree}{\href{http://saxonthree.nikoz.academy}{\textit{\airforcebluebf{http://saxonthree.nikoz.academy}}}}
\newcommand{\algebraone}{\href{http://algebra1.nikoz.academy}{\textit{\airforcebluebf{http://algebra1.nikoz.academy}}}}

%This section defines custom commands related to \enumerate and \itemize environments

\newcommand*\rightarrowitem{\item[\textcolor{black}{$\Rightarrow$}]}
\newcommand{\dur}[1]{\hfill\textit{(#1 min)}}
\newcommand{\question}{\purplebf{Question of the Day! }}
\newcommand{\questionitem}[1]{\item \purplebf{Question of the Day! } \dur{#1}}
\newcommand{\whiteboardskills}[1]{\item \greenbf{Whiteboards } Skills Development \dur{#1}}

\newcommand{\partab}{\par\vspace*{4pt}\hspace*{15pt}}


%This section defines \rcancel - cancel with red line

\newcommand{\rcancel}[1]{%
	\tikz[baseline=(tocancel.base)]{
		\node[inner sep=0pt,outer sep=0pt] (tocancel) {#1};
		\draw[red] (tocancel.south west) -- (tocancel.north east);
	}%
}

%This section defines command to shift line tags in \align and related environments
\makeatletter 
\newcommand{\leqnomode}{\tagsleft@true}
\newcommand{\reqnomode}{\tagsleft@false}
\makeatother


%this section defines the \arc symbol
\makeatletter
\DeclareFontFamily{U}{tipa}{}
\DeclareFontShape{U}{tipa}{m}{n}{<->tipa10}{}
\newcommand{\arc@char}{{\usefont{U}{tipa}{m}{n}\symbol{62}}}%

\newcommand{\arc}[1]{\mathpalette\arc@arc{#1}}

\newcommand{\arc@arc}[2]{%
	\sbox0{$\m@th#1#2$}%
	\vbox{
		\hbox{\resizebox{\wd0}{\height}{\arc@char}}
		\nointerlineskip
		\box0
	}%
}
\makeatother

 %this section defines fancy \pi typesetting
\newlength\curht   
\def\defaultdimfrac{.98}
\def\defaultstartht{\baselineskip}
\newcommand\diminish[2][\defaultdimfrac]{%
	\curht=\defaultstartht\relax
	\def\dimfrac{#1}%
	\diminishhelpA{#2}%
}
\newcommand\diminishhelpA[1]{%
	\expandafter\diminishhelpB#1\relax%
}
\def\diminishhelpB#1#2\relax{%
	\scaleto{\strut#1}{\curht}%
	\curht=\dimfrac\curht\relax%
	\ifx\relax#2\relax\else\diminishhelpA{#2}\fi%
}
\def\pinum{3.14159265358979323846264338327950288419716939937510}

%\AtBeginSection[]
%{
%	\begin{frame} %
%		\frametitle{Table of Contents}
%		\tableofcontents[currentsection,
%		sectionstyle=show,
%		subsectionstyle=hide/hide,
%		]
%	\end{frame}
%}

%\AtBeginSubsection[]
%{
%	\begin{frame}<beamer>
	%	\begin{multicols}{2}
%			\tableofcontents[currentsection,
%			sectionstyle=show/hide,
%			subsectionstyle=show/shade,
%			hideothersubsections]
	%	\end{multicols}
%	\end{frame}
%}

\author{Niko Z.}
\title{Lesson Plans \& Materials
	\\
	\smallskip
	$Feb 13^{th}$ - $17^{th}$}

%\subtitle{}
\titlegraphic{\includegraphics[width=0.1\linewidth]{Images/woodstock}}

\institute{Woodstock School}

\begin{document}
	
	
\frame{\titlepage}
\logo{\includegraphics[width=0.08\linewidth]{Images/woodstock}}
%\color{blue}	
   \section[Mon]{Monday}
   \subsection[PA2/01]{Period 2 - Pre-Algebra 2/01}
   \begin{frame}[label=PA2_01]
   	\frametitle{Lesson 54 - Angle Relationships}

    CHange in Debian 2

   \begin{alertblock}{}
   	Most of you can move on straight to \bluebf{Tangled Web} games assigned on \mangahightext. \\
   \end{alertblock}
<<<<<<< HEAD
      CHANGE!!!!
=======

>>>>>>> 7099d88e41a32c18acc21dd6328836d7575295d9
      \begin{enumerate}
   	    \questionitem{5}
        \item \mangahightext
        \rightarrowitem Prodigi activities about angles - recycled Winter Homework
        \rightarrowitem \bluebf{Tangled Web} games
      \end{enumerate}
   	  \end{frame}

     \subsection[ALG]{Period 4 - Algebra One}
   	 \begin{frame}[label=ALG1]
   	 	\frametitle{Linear Inequalities Review}

      \begin{enumerate}
   	    \item \bluebf{Google} Classroom code \redbf{vir3of}
        \rightarrowitem Inequalities cumulative Practice
        \rightarrowitem Quiz Review PacK
      \end{enumerate}

      \begin{alertblock}{}
        Questions on the quiz will be similar
     \end{alertblock}
   	  \end{frame}

   	 \subsection[PA1/02]{Period 5 - Pre-Algebra 1/02}
   	    \begin{frame}[label=PA1_02]
   	    		\frametitle{Lesson 53 Ratio Word Problems}

        \begin{itemize}
     	    \item \airforcebluebf{Civic Duty} Investigation
      \end{itemize}
      \vspace{-20pt}
        \begin{center}
\includegraphics[width=0.5\linewidth]{Images/honda_civic}
\end{center}
           \vspace{-20pt}
         \begin{alertblock}{}
         	Worksheet is posted on your \bluebf{Google} classroom \redbf{m8v08j}.
         	If you loose the paper just write down \underline{complete} answers in your notebook. You should finish it as \kugreenbf{homework}.
         \end{alertblock}
   	  \end{frame}

   	 \subsection[PA1/01]{Period 6 - Pre-Algebra 1/01}
   	   \begin{frame}[label=PA1_01]
   	 	  	\frametitle{Lesson 53 Ratio Word Problems}

   	 	  	\begin{itemize}
   	 	  		\item \airforcebluebf{Civic Duty} Investigation
   	 	  	\end{itemize}
   	 	  	\vspace{-20pt}
   	 	  	\begin{center}
   	 	  		\includegraphics[width=0.5\linewidth]{Images/honda_civic}
   	 	  	\end{center}
   	 	  	\vspace{-20pt}
   	 	  	\begin{alertblock}{}
   	 	  		Worksheet is posted on your \bluebf{Google} classroom \redbf{8b6zylm}.
   	 	  		If you loose the paper just write down \underline{complete} answers in your notebook. You should finish it as \kugreenbf{homework}.
   	 	  	\end{alertblock}
   	 	  \end{frame}

   	 \section[Tue]{Tuesday}
   	  \subsection[PA1/02]{Period 2 - Pre-Algebra 1/02}
   	    \begin{frame}[label=PA1_02]
   	    		\frametitle{Lesson 53 Ratio Word Problems Cont.}

                    \begin{itemize}
                    	\item Finish \airforcebluebf{Civic Duty} Investigation
                    \end{itemize}
                    \vspace{-20pt}
                    \begin{center}
                    	\includegraphics[width=0.5\linewidth]{Images/honda_civic}
                    \end{center}
                    \vspace{-20pt}
   	  \end{frame}

\subsection[PA2/01]{Period 3 - Pre-Algebra 2/01}
   	 \begin{frame}[label=PA2_01]
   	      	\frametitle{Running, Rates and Conversion Investigation \\ Part I}

           \begin{enumerate}
   	        \item Rates Investigation
            \rightarrowitem Make teams of two \dur{2}
            \rightarrowitem Discuss questions on the first two pages \dur{12}
            \rightarrowitem Measurement and calculation
           \end{enumerate}

           	\begin{alertblock}{}
           		You need to show your calculation on paper, including long division. You may use the calculator to \greenbf{check} your work.
           		You can round all calculation to \redbf{ONE} decimal place. If you are rounding to one decimal place, how many decimals do you have to calculate to?
           	\end{alertblock}


   	  \end{frame}


   	 \subsection[PA1/01]{Period 4 - Pre-Algebra 1/01}
   	 \begin{frame}[label=PA1_01]
       		\frametitle{Lesson 53 Ratio Word Problems Cont.}

       		\begin{itemize}
       			\item Finish \airforcebluebf{Civic Duty} Investigation
       		\end{itemize}
       		\vspace{-20pt}
       		\begin{center}
       			\includegraphics[width=0.5\linewidth]{Images/honda_civic}
       		\end{center}
       		\vspace{-20pt}
       	\end{frame}


   	 \subsection[PA2/02]{Period 5 - Pre-Algebra 2/02}
   	 \begin{frame}[label=PA2_02]
   	 \frametitle{Lesson 54 - Angle Relationships}

   	 \begin{alertblock}{}
   	 	Most of you can move on straight to \bluebf{Tangled Web} games assigned on \mangahightext. \\
   	 \end{alertblock}

   	 \begin{enumerate}
   	 	\questionitem{5}
   	 	\item \mangahightext
   	 	\rightarrowitem Prodigi activities about angles - recycled Winter Homework
   	 	\rightarrowitem \bluebf{Tangled Web} games
   	 \end{enumerate}
   	  \end{frame}

\subsection[ALG1]{Period 6 - Algebra One}
   	   \begin{frame}[label=ALG1]
   	   \frametitle{Quiz Time!}

           \begin{center}
\includegraphics[width=0.7\linewidth]{Images/keep_calm_have_a_quiz}
\end{center}

   	   \end{frame}

   	 \section[Wed]{Wednesday}
   	 \subsection[PA2/02]{Period 3 - Pre-Algebra 2/02}
      \begin{frame}[label=PA2_02]
       	\frametitle{Running, Rates and Conversion Investigation \\ Part I}

        	\begin{enumerate}
              \item Rates Investigation
               \rightarrowitem Make teams of two \dur{2}
               \rightarrowitem Discuss questions on the first two pages \dur{12}
               \rightarrowitem Measurement and calculation
             \end{enumerate}

\begin{alertblock}{}
	You need to show your calculation on paper, including long division. You may use the calculator to \greenbf{check} your work.
	You can round all calculation to \redbf{ONE} decimal place. If you are rounding to one decimal place, how many decimals do you have to calculate to?
\end{alertblock}

      \end{frame}

 \subsection[PA1/02]{Period 5 - Pre-Algebra 1/02}
      \begin{frame}[label=PA1_02]
        	\frametitle{Lesson 53 Ratio Word Problems Cont.}

       	\begin{itemize}
       		\item Finish \airforcebluebf{Civic Duty} Investigation
       		\\ This time \redbf{REALLY} finish \dots
       	\end{itemize}
       	\vspace{-20pt}
       	\begin{center}
       		\includegraphics[width=0.5\linewidth]{Images/honda_civic}
       	\end{center}
       	\vspace{-20pt}
       \end{frame}

   	 \section[Thu]{Thursday}
 \subsection[PA1/01]{Period 1 - Pre-Algebra 1/01}
 \begin{frame}[label=PA1_01]
  	       	\frametitle{Lesson 53 Ratio Word Problems Cont.}

  	       	\begin{itemize}
  	       		\item Finish \airforcebluebf{Civic Duty} Investigation
  	       		\\ This time \redbf{REALLY} finish \dots
  	       	\end{itemize}
  	       	\vspace{-20pt}
  	       	\begin{center}
  	       		\includegraphics[width=0.5\linewidth]{Images/honda_civic}
  	       	\end{center}
  	       	\vspace{-20pt}
  	       \end{frame}

  \subsection[PA1/02]{Period 3 - Pre-Algebra 1/02}
	 \begin{frame}[label=PA1_02]
	 	       	\frametitle{Lesson 53 Ratio Word Problems Cont.}

	 	       	\begin{itemize}
	 	       		\item Finish \airforcebluebf{Civic Duty} Investigation
	 	       		\\ This time \redbf{REALLY} finish \dots
	 	       	\end{itemize}
	 	       	\vspace{-20pt}
	 	       	\begin{center}
	 	       		\includegraphics[width=0.5\linewidth]{Images/honda_civic}
	 	       	\end{center}
	 	       	\vspace{-20pt}
	 	       \end{frame}

 \subsection[ALG]{Period 4 - Algebra One}
 \begin{frame}[label=ALG1]
 	\frametitle{Inequalities Quiz Review}

        \begin{enumerate}
   	   	   \item Quiz Review \dur{25}
   	   	   \item \purplebf{BONUS} Project \dur{20}
   	     \end{enumerate}

 	 	\end{frame}


 \subsection[PA2/01]{Period 5 - Pre-Algebra 2/01}
   \begin{frame}[label=PA2_01]
 	\frametitle{Running, Rates and Conversion Investigation \\ Part II}

        \begin{enumerate}
   	   	   \item Unit Conversion \greenbf{examples} \dur{10}
   	   	   \rightarrowitem Unit Conversion \redbf{Worksheet} \dur{30}
   	     \end{enumerate}
   	\vspace{-12pt}
   	\begin{center}
   		\includegraphics[width=0.7\linewidth]{Images/running}
   	\end{center}
   	\vspace{-20pt}
 	\end{frame}

  \subsection[PA2/02]{Period 6 - Pre-Algebra 2/02}
  \begin{frame}[label=PA2_02]
  	\frametitle{Running, Rates and Conversion Investigation \\ Part II}

       \begin{enumerate}
   	   	    \item Unit Conversion \greenbf{examples} \dur{10}
   	   	    \rightarrowitem Unit Conversion \redbf{Worksheet} \dur{30}
   	     \end{enumerate}
   	  	\vspace{-12pt}
   	  	\begin{center}
   	  		\includegraphics[width=0.7\linewidth]{Images/running}
   	  	\end{center}
   	  	\vspace{-20pt}

    \end{frame}

   	 \section[Fri]{Friday}
         \subsection[PA2/02]{Period 1 - Pre-Algebra 2/02}
         \begin{frame}[label=PA2_02]
           	\frametitle{Finish Running, Rates and Conversion Investigation}

          \begin{enumerate}
   	   	   \item \bluebf{Google} classroom \redbf{kxjlo7}
   	   	   \rightarrowitem Open \emph{Frequency Table Worksheet} and put your average rate in the appropriate place in the table
   	   	   \rightarrowitem Use the data in the table to construct a \bluebf{bar graph}
   	   	   \item \mangahighlogo
   	     \end{enumerate}

         \end{frame}
     \subsection[ALG]{Period 3 - Algebra One}
     \begin{frame}[label=ALG1]
     	\frametitle{Chapter 6 Systems of Linear Equations}

           \begin{enumerate}
   	   	   \item Introduction - Spring Relationships \dur{10}
   	   	   \item \orangebf{VIDEO} Gold filling scam \dur{10}
   	   	   \item \greenbf{Examples} {Solve linear equations by graphing} \dur{15}
   	   	   \item Choice of project or \Texbook work
   	     \end{enumerate}

      \end{frame}
     \subsection[PA1/01]{Period 5 - Pre-Algebra 1/01}
     \begin{frame}[label=PA1_01]
     	    	\frametitle{Rates \& Ratios Contd.}

     	    	\begin{itemize}
     	    		\item Finish \& Discuss \airforcebluebf{Civic Duty} Investigation
     	    		\rightarrowitem \mangahighlogo
     	        \end{itemize}
     	    	\vspace{-20pt}
     	    	\begin{center}
     	    		\includegraphics[width=0.5\linewidth]{Images/honda_civic}
     	    	\end{center}
     	    	\vspace{-20pt}
     	    \end{frame}

      \subsection[PA2/01]{Period 6 - Pre-Algebra 2/01}
      \begin{frame}[label=PA2_01]
      	   	\frametitle{Finish Running, Rates and Conversion Investigation}

      	   	\begin{enumerate}
      	   		\item \bluebf{Google} classroom \redbf{u61hzgg}
      	   		\rightarrowitem Open \emph{Frequency Table Worksheet} and put your average rate in the appropriate place in the table
      	   		\rightarrowitem Use the data in the table to construct a \bluebf{bar graph}
      	   		\item \mangahighlogo
      	   	\end{enumerate}

      \end{frame}

%\section[]{Templates}
   	 
   	  \subsection[]{Enumerate Auto Advance \& Alert}
   	  \begin{frame}
   	  	\frametitle{Enumerate Auto Advance \& Alert}
   	  	
   	  	\begin{enumerate}[<+- | alert@+>]
   	  		\item Textbook height in centimeters?
   	  		\item Textbook width in milimeters?
% * <nikotheteacher@gmail.com> 2016-11-26T16:21:20.335Z:
%
% ^.
% * <nikotheteacher@gmail.com> 2016-11-26T16:21:18.099Z:
%
% ^.
% * <nikotheteacher@gmail.com> 2016-11-26T16:20:28.190Z:
%
% ^.
% * <nikotheteacher@gmail.com> 2016-11-26T16:20:26.351Z:
%
% ^.
   	  		\item Notebook width in meters?
   	  		\item Whiteboard width in milimeters?
   	  	\end{enumerate}
   	  \end{frame}
   	  
   	 \subsection[]{Align Env. with Highlighting}
   	   \begin{frame}
   	   	\setcounter{equation}{0}
   	   	\begin{align}
   	   	\uncover<1->{\alert<1>{y} &= \alert<1>{mx + b} \\}
   	   	\uncover<2->{\alert<2>{m} &= \alert<2>{\dfrac{y_1 - y_2}{x_1 - x_2}}\\}
   	   	\uncover<3->{\alert<3>{m(x_1 - x_2)}&= \alert<3>{(y_1 - y_2)}\\}
   	   	\uncover<4->{\alert<4>{(y_1 - y_2)} &=  \alert<4>{m(x_1 - x_2)}\\}
   	   	\uncover<5->{\alert<5>{(y - y_2)} &=  \alert<5>{m(y - x_2)}\\}
   	   	\uncover<6->{\alert<6>{(y - y_1)} &=  \alert<6>{m(y - x_1)}}
   	   	\notag
   	   	\end{align}
   	   	\vskip-1.5em
   	   \end{frame}
   	   
   	  \subsection[]{Column Environment with equation on the left side}
   	  \begin{frame}
   	  	\frametitle{Column Environment with equation on the left side}
   	  	Some question.
   	  	
   	  	\begin{columns}
   	  		\begin{column}{0.4\textwidth}
   	  			\setcounter{equation}{0}
   	  			\begin{align}
   	  			\uncover<2->{\alert<2>{c} &= \alert<2>{\pi \times d} \\}
   	  			\uncover<3->{\alert<3>{c} & \alert<3>{\approx \dfrac{22}{7} \times 35 in}\\}
   	  			\uncover<4->{\alert<4>{c} & \alert<4>{\approx \dfrac{22}{\cancelto{1}{7}} \times \cancelto{5}{35} in}\\}
   	  			\uncover<5->{\alert<5>{c}& \alert<5>{\approx 110 in}\\}
   	  			\notag
   	  			\end{align}    	 	
   	  		\end{column}
   	  		\begin{column}{0.7\textwidth}    	
   	  		
   	  		\end{column}
   	  	\end{columns}
   	  	
   	  \end{frame}
   	  \subsection[]{Number Line I}
   	   \begin{frame}[label=PA2_01]
   	   	\frametitle{Number Line}
   	   	\NumberLine*[
   	   	dot=blue,
   	   	dot opacity=.75,
   	   	fill=blue,
   	   	fraction=1,
   	   	ticks above=false,
   	   	min=0,
   	   	number to=none,
   	   	h scale=3,
   	   	mark at={14/6,0.7,0.9},
   	   	]
   	   	\\
   	     \end{frame}   
   	     
   	   \subsection[]{Number Line II}
   	    \begin{frame}
   	      \begin{tikzpicture}
   	      \begin{axis}[
   	      axis x line=middle,
   	      % we don't need a y axis line ...
   	      axis y line=none,
   	      % ... and thus there is no need for much `height' of the axis
   	      height=50pt,
   	      % but `height' also changes `width' which is restored here
   	      width=\axisdefaultwidth,
   	      xmin=-3,
   	      xmax=3,
   	      ]
   	      \addplot coordinates {
   	      	(0.5,0) (0.7,0) (0.9,0)
   	      };
   	      \end{axis}
   	      \end{tikzpicture}   
   	     \end{frame}  	   
   	   
   	   \subsection[]{Fill in the blanks}
   	   \begin{frame}
   	   	 \frametitle{Fill in the blanks}
   	   	 \vspace*{-10pt}
   	   	\only<2>{\fillbhideanswerfalse}
   	   	\setcounter{equation}{0} 
   	   	\begin{columns}[t]
   	   		\begin{column}{0.5\textwidth}
   	   			\begin{align}
   	   			\dfrac{5}{6}\times\dfrac{\fillb{6}}{\fillb{5}} &= 1 \\
   	   			\dfrac{5}{6}\times\dfrac{\fillb{6}}{\fillb{5}} &= 1 \\
   	   			\dfrac{5}{6}\times\dfrac{\fillb{6}}{\fillb{5}} &= 1 \\
   	   			\dfrac{5}{6}\times\dfrac{\fillb{6}}{\fillb{5}} &= 1 \\
   	   			\dfrac{5}{6}\times\dfrac{\fillb{6}}{\fillb{5}} &= 1
   	   			\end{align}
   	   		\end{column}
   	   		\begin{column}{0.5\textwidth}
   	   			\begin{align}
   	   			\dfrac{5}{6}\times\dfrac{\fillb{6}}{\fillb{5}} &= 1 \\
   	   			\dfrac{5}{6}\times\dfrac{\fillb{6}}{\fillb{5}} &= 1 \\
   	   			\dfrac{5}{6}\times\dfrac{\fillb{6}}{\fillb{5}} &= 1 \\
   	   			\dfrac{5}{6}\times\dfrac{\fillb{6}}{\fillb{5}} &= 1 \\
   	   			\dfrac{5}{6}\times\dfrac{\fillb{6}}{\fillb{5}} &= 1
   	   			\end{align}
   	   		\end{column}
   	   	\end{columns}
   	   \end{frame}
   	   
   	     \subsection[]{Probability Spinner}
   	     \frametitle{Probability Spinner}
   	     \begin{frame}
   	     	\begin{tikzpicture}
   	     	\pie [sum=10, explode=0.1, text=inside,color={blue!70, green!80, red, yellow},
   	     	before number=\phantom,after number=]
   	     	{1/A, 2/, 3/Number, 4/Labels are optional}
   	     	\node [scale=3, rotate=75](note) at (0,0) {\Huge $\twoheadrightarrow$};
   	     	\end{tikzpicture}
   	     	
   	     \end{frame}	
   	   
   	   \subsection[]{Pie Chart}
   	     \frametitle{Pie Chart}
   	   \begin{frame}
   	   	 \begin{tikzpicture}
             	  \pie[sum = 10, text = legend,  explode = 0.05]{3/A , 2/B, 2/C, 3/D}
   	   	 \end{tikzpicture}
   	  
   	   \end{frame}
   	   
   	   \subsection[]{Bar Graph}
   	    \begin{frame}
   	    	\frametitle{Bar Graph}
   	    	    	\begin{tikzpicture}
   	    	    	\begin{axis}
   	    	    	[
   	    	    	symbolic x coords={0-30, 31-60, 61-90,  91-120,  121-150, },
   	    	    	xtick=data,
   	    	    	axis lines* = left,
   	    	    	ymajorgrids= true,
   	    	    %	enlarge x limits=0.02,
   	    	        ybar,
   	    	 %       ybar interval=0.5,
   	    	        xmajorgrids= false,
   	    	        bar width=8pt, 
   	    	        width=0.85\textwidth,
   	    	    	ylabel= Y axis label, 
   	    	    	xlabel= \bluebf{X axis label (unit)}
   	    	    	]
   	    	    	\addplot[fill=purple] coordinates {
   	    	    		(0-30,  4)
   	    	    		(31-60,  3)
   	    	    		(61-90,   6)
   	    	    		(91-120,   3)
   	    	    		(121-150,   4)
   	    	    	};
   	    	    		\addplot[fill=blue] coordinates {
   	    	    			(0-30,  6)
   	    	    			(31-60,  3)
   	    	    			(61-90,   6)
   	    	    			(91-120,   3)
   	    	    			(121-150,   4)
   	    	    		};
   	    	    	\end{axis}
   	    	    	\end{tikzpicture}
   	    \end{frame}
   	    
   	    \subsection[]{Function Graph}
   	    \begin{frame}
   	    	\frametitle{Funtion Graph}
   	    	\begin{tikzpicture}
   	    	\begin{axis}[
   	    	axis lines = center,
   	    	xlabel = $x$,
   	    	xlabel style={at=(current axis.right of origin), anchor=west},
   	    	ylabel = {$y$},
   	    	ylabel style={at=(current axis.above origin), anchor=south},
   	    	%legend style = { at = {(1.0,1.0)}},
   	    	%  xtick={-5,-4,-3,-2,-1,0,1,2,3,4,5,},
   	    	% ytick={-5,-4,-3,-2,0,1,2,3,4,5},
   	    	%	enlarge y limits={rel=0.07}, 
   	    	%enlarge x limits={rel=0.07}, 
   	    	legend pos=north east,
   	    	ymajorgrids=true,
   	    	xmajorgrids=true,
   	    	grid style=  gray!20,
   	    	]
   	    	%Below the red parabola is defined
   	    	\addplot [
   	    	domain=-4:4, 
   	    	samples=100, 
   	    	color=red,
   	    	]
   	    	{2*x + 1};
   	    	\addlegendentry{$2x + 1$}
   	    	\addplot [
   	    	domain=-5:5, 
   	    	samples=100, 
   	    	color=blue,
   	    	]
   	    	{2*x + 4};
   	    	\addlegendentry{$2x + 4$}
   	    	\end{axis}
   	    	\end{tikzpicture}
   	    \end{frame}
   	    
   	    
   	    
   	    \subsection[]{Multicolored Grid}
   	    \begin{frame}
   	    	\frametitle{Fractions and Ratio}   	     	
   	    	
   	    	\begin{columns}
   	    		\begin{column}{0.5\textwidth}
   	    			What is the fraction of green squares?
   	    			\\
   	    			\smallskip
   	    			\uncover<2->{$\dfrac{3}{16}$}
   	    			\\
   	    			\bigskip
   	    			What is the ratio of red, blue and green squares?
   	    			\\
   	    			\smallskip
   	    			\uncover<3->{4:9:3}
   	    			
   	    		\end{column}
   	    		\begin{column}{0.5\textwidth}
   	    			
   	    			\begin{tikzpicture}[every node/.style={minimum size=1cm-\pgflinewidth, outer sep=3pt}]
   	    			\draw[step=1cm,color=black] (-1,-1) grid (3,3);
   	    			\node[fill=red!80] at (-0.5,-0.5) {};
   	    			\node[fill=red!80] at (0.5,-0.5) {};
   	    			\node[fill=red!80] at (1.5,-0.5) {};
   	    			\node[fill=red!80] at (2.5,-0.5) {};
   	    			\node[fill=green!80] at (-0.5,0.5) {};
   	    			\node[fill=green!80] at (0.5,0.5) {};
   	    			\node[fill=green!80] at (1.5,0.5) {};
   	    			\node[fill=green!80] at (2.5,0.5) {};
   	    			\node[fill=blue!80] at (-0.5,1.5) {};
   	    			\node[fill=blue!80] at (0.5,1.5) {};
   	    			\node[fill=blue!80] at (1.5,1.5) {};
   	    			\node[fill=blue!80] at (2.5,1.5) {};
   	    			\node[fill=blue!80] at (-0.5,2.5) {};
   	    			\node[fill=blue!80] at (0.5,2.5) {};
   	    			\node[fill=blue!80] at (1.5,2.5) {};
   	    			\node[fill=blue!80] at (2.5,2.5) {};
   	    			\end{tikzpicture}
   	    			
   	    		\end{column}
   	    		
   	    	\end{columns}
   	    \end{frame}
   	    
   	    \subsection[]{Co-ordinate Drawing - Labelled Star}
   	    	\begin{frame}
   	    		\frametitle{Co-ordinate Drawing}
   	    		
   	    		\begin{center}
   	    			\begin{tikzpicture}
   	    			\begin{axis}[
   	    			axis lines = center, %left, center, right
   	    			%	title={Stone in Free-fall},
   	    			xlabel={x},
   	    			xlabel style={at=(current axis.right of origin), anchor=west},
   	    			ylabel={y},
   	    			ylabel style={at=(current axis.above origin), anchor=south},
   	    			xmin=-6, xmax=6,
   	    			ymin=-6, ymax=6,
   	    			xtick={-6,-5,-4,-3,-2,-1,0,1,2,3,4,5,6},
   	    			ytick={-6,-5,-4,-3,-2,-1,0,1,2,3,4,5,6},
   	    			enlarge y limits={rel=0.07}, 
   	    			enlarge x limits={rel=0.07}, 
   	    			legend pos=north east,
   	    			ymajorgrids=true,
   	    			xmajorgrids=true,
   	    			grid style=  gray!20,
   	    			]
   	    			
   	    			\addplot [
   	    			color=blue,
   	    			mark=*, % * is a filled circle, other options are square
   	    			%only marks,
   	    			]
   	    			coordinates {
   	    				(-5,1)(3,-4)(-5,1)(5,1)(-3,-4)(0,4)(3,-4)(-5,1)
   	    			}
   	    			%     \node[anchor=north] at (axis cs:2,1) {$f(\theta_A, x)$};
   	    			%	  node[pin={$(2,-1)$}] at (axis cs: 2,1) {}
   	    			node [anchor=south] at (axis cs: -5,1) {$(-5,1)$}
   	    			node [anchor=south] at (axis cs: 5,1) {$(5,1)$}
   	    			node [anchor=west] at (axis cs: 0,4) {$(0,4)$}
   	    			node [anchor=north] at (axis cs: -3,-4) {$(-3,-4)$}
   	    			node [anchor=north] at (axis cs: 3,-4) {$(3,-4)$}
   	    			;
   	    			
   	    			\end{axis}
   	    			\end{tikzpicture} 
   	    		\end{center}
   	    		
   	    	\end{frame}
   	    
   	     \subsection[]{Triangle}
   	     \begin{frame}
   	     	\frametitle{Triangle}
   	        \begin{tikzpicture}[thick,color=crimson]
   	        \coordinate (O) at (0,0);
   	        \coordinate (A) at (4,0);
   	        \coordinate (B) at (2.4,2.2);
   	        \draw (O)--(A)--(B)--cycle;
   	        
   	        \tkzLabelSegment[below=2pt](O,A){$2(x + 4)$ cm}
   	        \tkzLabelSegment[left=6pt](O,B){$3x$ cm}
   	        \tkzLabelSegment[above right=2pt](A,B){$2x + 1$ cm}
   	        \end{tikzpicture} 
   	     \end{frame}
   	     \subsubsection{Isosceles Triangle}
   	     \begin{frame}
   	     	\frametitle{Isosceles Triangle}
   	     	
   	     	\begin{tikzpicture}[scale=1]
   	     	\tkzDefPoints{0/0/O,2/2/A,4/0/B,6/2/C}
   	     	\tkzDrawSegments(O,A A,B)
   	     	\tkzDrawPoints(O,A,B)
   	     	\tkzDrawLine(O,B)
   	     	\tkzMarkSegments[mark=|,size=4pt](O,A A,B)
   	     	\end{tikzpicture}
   	     	
   	     \end{frame}
   	     
   	     \subsection[]{Rectangle}
   	         \begin{frame}
   	         	\frametitle{Rectangle}
   	         	\begin{tikzpicture}[thick,color=crimson]
   	         	\coordinate (O) at (0,0);
   	         	\coordinate (A) at (4,0);
   	         	\coordinate (B) at (4,4);
   	         	\coordinate (C) at (0,4);
   	         	\draw (O)--(A)--(B)--(C)--cycle;
   	         	
   	         	\tkzLabelSegment[below=2pt](O,A){$2(x + 4)$ cm}
   	         	\tkzLabelSegment[left=6pt](O,B){$3x$ cm}
   	         	\tkzLabelSegment[above right=2pt](A,B){$2x + 1$ cm}
   	         	\end{tikzpicture} 
   	         \end{frame}
   	         
   	         \subsection[]{Circle}
   	           \begin{frame}
   	           	  \frametitle{Circle Ring}
   	            \begin{tikzpicture}[scale=2]
   	            \tkzDefPoint(0,0){O}
   	            \tkzDefPoint(1,0){A}
   	             \tkzDefPoint(-1.5,0){B}
   	             \tkzDefPoint(-1.6,0){r_1=3}
   	              \tkzDefPoint(0,1){C}
   	               \tkzDefPoint(0,0.6){r_2=2}
   	             \tkzDrawCircle[color=blue, fill = blue](O,B)
   	            \tkzDrawCircle[color=red, fill = red](O,A)
   	            \tkzDrawSegments[color=black,ultra thick](C,O O,B)
   	            \tkzLabelPoints[color=green, ultra thick](r_1=3)
   	              \tkzLabelPoints[color=green, ultra thick](r_2=2)
   	           
   	            \end{tikzpicture} 
   	             \end{frame}
   	             
   	             \begin{frame}
   	             	\frametitle{Semi Circle}
   	             	\begin{tikzpicture}[scale=.75]
   	             	\tkzInit[ymax=8,xmax=8]
   	             	%	\tkzClip[space=.25]
   	             	\tkzDefPoint(0,0){A}
   	             	\tkzDefPoint(8,0){B} 
   	             	\tkzDefPoint(4,0){O}
   	             	\tkzDrawSector[fill=yellow](O,B)(A)
   	             	\tkzDrawPoints(A,B,O)
   	             	\tkzLabelPoints(A,B,O)
   	             	\tkzLabelSegment[below=2pt](A,O){$\overline{A0}=4$}
   	             	\end{tikzpicture}
   	             \end{frame}
   	         
   	            \begin{frame}
   	              \frametitle{Circle Arc I}
   	              
   	             \begin{tikzpicture}
   	             \tkzInit[ymin=-2.25,ymax=2.25,xmin=-2.25,xmax=2.25]
   	             \tkzDefPoint(0,0){O}
   	             \tkzDefPoint(2,0){N}
   	             \tkzDefPointBy[rotation=center O angle 20](N)
   	             \tkzGetPoint{M}
   	             \tkzDefPointBy[rotation=center O angle -20](N)
   	             \tkzGetPoint{P}
   	             \tkzDefPointBy[rotation=center O angle 125](N)
   	             \tkzGetPoint{P’}
   	             \tkzLabelCircle[above=4pt](O,N)(120){Label}
   	             \tkzDrawCircle(O,M)
   	             \tkzFillCircle[color=blue!20,opacity=.4](O,M)
   	             \tkzLabelCircle[R,draw,fill=orange!30,%
   	             text width=2cm,text centered](O,3 cm)(-60)%
   	             {Label}
   	             \tkzDrawSegment[dashed](O,P)
   	             \tkzDrawSegment[dashed](O,M)
   	             \tkzDrawPoints(M,P)\tkzLabelPoints[right](M,P)
   	             \end{tikzpicture}
   	             
   	            \end{frame}
   	         
   	             \begin{frame}
   	             	\frametitle{Circle Chrod and Arc}
   	             	
   	             	\begin{tikzpicture}[scale=1.5]
   	             	\tkzDefPoint(0,0){O}
   	             	\tkzDefPoint(2,0.5){A}
   	             	\tkzDefPointBy[rotation= center O angle 30](A)
   	             	\tkzGetPoint{B}
   	             	\tkzDrawArc[color=blue, thick](O,A)(B)
   	             	\tkzDrawArc(O,B)(A)
   	             	\tkzDrawLines[add = 0 and .5](O,A O,B)
   	             	\tkzLabelLine[above, rotate=48](O,B){$\overline{OB}=2m$}
   	             	\tkzDrawPoints(O,A,B)
   	             	\tkzLabelPoints[below](O,A,B)
   	             	\tkzLabelAngle[color=blue, rotate = 18](A,O,B){$\measuredangle = 30\degree$}
   	             	\tkzFillSector[rotate,color=orange!70,opacity=0.2](O,B)(330)
   	             	\tkzFillSector[rotate,color=blue!70,opacity=0.4](O,A)(30)
   	             	\end{tikzpicture}
   	             	
   	             \end{frame}
                 
 \subsection[]{Probability Trees}
   \subsubsection{Probability Tree - Two Coins}
   	\begin{frame}{Probability Tree - Two Coins}
      \begin{tikzpicture}[grow=right, sloped]
 		\tikzstyle{level 1}=[level distance=3.5cm, sibling distance=3.5cm]
		\tikzstyle{level 2}=[level distance=3.5cm, sibling distance=2cm]
		\tikzstyle{toss} = [text width=4em, text centered]
		\tikzstyle{end} = [circle, minimum width=3pt,fill, inner sep=0pt]
			\node[toss] {Coin 1}
				child<2-> {
				   node[toss] {Tails}
						child<3-> {
					    node[toss,label=right:{label}] {Head}
                              }
				       child<3-> {
	    			    node[toss] {Tails}
							}
					       }
				child<4-> {
					node[toss] {Head}
						child<5-> {
							node[toss] {Tails}
									}
						 child<6-> {
						    node[toss] {This could be an image}
								}		
						}
 					;				
		\end{tikzpicture}                 
                 \end{frame}
           
           
           \subsubsection{Probability Trees - Two Coins Downward}
           \begin{frame}{Probability Tree - Two Coins $\Downarrow$}
           	\begin{tikzpicture}[grow=down, sloped]
           	\tikzstyle{level 1}=[level distance=1.5cm, sibling distance=4cm]
           	\tikzstyle{level 2}=[level distance=1.5cm, sibling distance=2.5cm]
           	\tikzstyle{toss} = [text width=4em, text centered]
           	\tikzstyle{end} = [circle, minimum width=3pt,fill, inner sep=0pt]
           	\node[toss] {\includegraphics[width=0.4\linewidth]{Images/coin}}
           	child<2-> {
           		node[toss] {Tails}
           		child<6-> {
           			node[toss,label=below:{\bluebf{TH}}] {Head}
           		}
           		child<5-> {
           			node[toss,label=below:{\bluebf{TT}}] {Tails}
           		}
           	}
           	child<2-> {
           		node[toss] {Head}
           		child<4-> {
           			node[toss,label=below:{\redbf{HT}}] {Tails}
           		}
           		child<3-> {
           			node[toss,label=below:{\redbf{HH}}] {Head}
           		}		
           	}
           	;				
           	\end{tikzpicture}                 
           \end{frame}
           
           \subsubsection{Probability Tree - Die and Coin Downward}               
           \begin{frame}{Probability Tree - Die and Coin}
           	%	\begin{center}
           	\hspace*{-20pt}	\begin{tikzpicture}[grow=down, sloped]
           	\tikzstyle{level 1}=[level distance=1.5cm, sibling distance=1.8cm]
           	\tikzstyle{level 2}=[level distance=1.5cm, sibling distance=1cm]
           	\tikzstyle{toss} = [text width=4em, text centered]
           	\tikzstyle{end} = [circle, minimum width=3pt,fill, inner sep=0pt]
           	\node[toss] {\includegraphics[width=0.4\linewidth]{Images/dice}}
           	child<2-> {
           		node[toss] {1}
           		child<3-> {
           			node[toss,label=below:{\bluebf{1T}}] {T}
           		}
           		child<3-> {
           			node[toss,label=below:{\bluebf{1H}}] {H}
           		}
           	}
           	child<2-> {
           		node[toss] {2}
           		child<4-> {
           			node[toss,label=below:{\redbf{2T}}] {T}
           		}
           		child<4-> {
           			node[toss,label=below:{\redbf{2H}}] {H}
           		}		
           	}
           	child<2-> {
           		node[toss] {3}
           		child<5-> {
           			node[toss,label=below:{\bluebf{3T}}] {T}
           		}
           		child<5-> {
           			node[toss,label=below:{\bluebf{3H}}] {H}
           		}
           	}
           	child<2-> {
           		node[toss] {4}
           		child<5-> {
           			node[toss,label=below:{\redbf{4T}}] {T}
           		}
           		child<5-> {
           			node[toss,label=below:{\redbf{4H}}] {H}
           		}		
           	}
           	child<2-> {
           		node[toss] {5}
           		child<5-> {
           			node[toss,label=below:{\bluebf{5T}}] {T}
           		}
           		child<5-> {
           			node[toss,label=below:{\bluebf{5H}}] {H}
           		}
           	}
           	child<2-> {
           		node[toss] {6}
           		child<5-> {
           			node[toss,label=below:{\redbf{6T}}] {T}
           		}
           		child<5-> {
           			node[toss,label=below:{\redbf{6H}}] {H}
           		}		
           	}
           	;				
           	\end{tikzpicture}    
           	%	\end{center}             
           \end{frame}
           \subsubsection{Probability Tree - Coin \& Die Downward}               
           \begin{frame}{Probability Tree - Coin \& Die}
           	%	\begin{center}
           	\hspace*{-30pt}
           	\begin{tikzpicture}[grow=down, sloped]
           	\tikzstyle{level 1}=[level distance=1.2cm, sibling distance=6cm]
           	\tikzstyle{level 2}=[level distance=1.8cm, sibling distance=1cm]
           	\tikzstyle{toss} = [text width=4em, text centered]
           	\tikzstyle{end} = [circle, minimum width=3pt,fill, inner sep=0pt]
           	\node[toss] {\includegraphics[width=0.4\linewidth]{Images/coin}}
           	child<2-> {
           		node[toss] {Head}
           		child<3-> {
           			node[toss,label=below:{\bluebf{H1}}] {1}
           		}
           		child<3-> {
           			node[toss,label=below:{\bluebf{H2}}] {2}
           		}
           		child<3-> {
           			node[toss,label=below:{\bluebf{H3}}] {3}
           		}
           		child<3-> {
           			node[toss,label=below:{\bluebf{H4}}] {4}
           		}
           		child<3-> {
           			node[toss,label=below:{\bluebf{H5}}] {5}
           		}
           		child<3-> {
           			node[toss,label=below:{\bluebf{H6}}] {6}
           		}
           	}
           	child<2-> {
           		node[toss] {Tail}
           		child<4-> {
           			node[toss,label=below:{\redbf{T1}}] {1}
           		}
           		child<4-> {
           			node[toss,label=below:{\redbf{T2}}] {2}
           		}	
           		child<4-> {
           			node[toss,label=below:{\redbf{T3}}] {3}
           		}
           		child<4-> {
           			node[toss,label=below:{\redbf{T4}}] {4}
           		}
           		child<4-> {
           			node[toss,label=below:{\redbf{T5}}] {5}
           		}
           		child<4-> {
           			node[toss,label=below:{\redbf{T6}}] {6}
           		}	
           	}
           	;				
           	\end{tikzpicture}    
           	%	\end{center}             
           \end{frame}
                 
   	      	     \subsection[]{Sample Space - Pair of Dice}
   	      	     \begin{frame}
   	      	     	\frametitle{Sample Space - Pair of Dice B \& W}
   	      	     	$	\begin{array}{ccccccc}
   	      	     	& {\large \bluebf{\epsdice{1}}} & {\large \bluebf{\epsdice{2}}}  & {\large \bluebf{\epsdice{3}}}  & {\large \bluebf{\epsdice{4}}}  &  {\large \bluebf{\epsdice{5}}} &  {\large \bluebf{\epsdice{6}}} \\ 
   	      	     	{\large {\epsdice[black]{1}}}  	& 2 & 3 & 4 & 5 & 6 & 7  \\ 
   	      	     	{\large {\epsdice[black]{2}}}    & 3 & 4 & 5 & 6 & 7 & 8  \\ 
   	      	     	{\large {\epsdice[black]{3}}}    & 4 & 5 & 6 & 7 & 8 & 9  \\ 
   	      	     	{\large {\epsdice[black]{4}}}    & 5 & 6 & 7 & 8 & 9 & 10 \\ 
   	      	     	{\large {\epsdice[black]{5}}}   	& 6 & 7 & 8 & 9 & 10 & 11 \\ 
   	      	     	{\large {\epsdice[black]{6}}}   	& 7 & 8 & 9 & 10 & 11 & 12
   	      	     	\end{array} $
   	      	     \end{frame}
   	      	     
   	      	     \begin{frame}
   	      	     	\frametitle{Sample Space - Pair of Dice Blue \& Red}
   	      	     	
   	      	     	$	\begin{array}{ccccccc}
   	      	     	& \bluebf{\Cube{1}}   & \bluebf{\Cube{2}} & \bluebf{\Cube{3}}  & \bluebf{\Cube{4}}  &  \bluebf{\Cube{5}} &  \bluebf{\Cube{6}} \\ 
   	      	     	\redbf{\Cube{1}}     		& 2 & 3 & 4 & 5 & 6 & 7 \\ 
   	      	     	\redbf{\Cube{2}}       	& 3 & 4 & 5 & 6 & 7 & 8 \\ 
   	      	     	\redbf{\Cube{3}}       	& 4 & 5 & 6 & 7 & 8 & 9 \\ 
   	      	     	\redbf{\Cube{4}}      	& 5 & 6 & 7 & 8 & 9 & 10 \\ 
   	      	     	\redbf{\Cube{5}}     		& 6 & 7 & 8 & 9 & 10 & 11 \\ 
   	      	     	\redbf{\Cube{6}}   		& 7 & 8 & 9 & 10 & 11 & 12
   	      	     	\end{array} $
   	      	     	
   	      	     \end{frame}
   	      	     
   	      	       \subsection[]{Sample Space - A die and Spinner}
   	      	       
   	      	         \begin{frame}   	      	       
   	      	           \frametitle{Probability - A die and Spinner}
   	      	           
   	      	           A 6 number die is tossed and the spinner spun at the same time. What is the probability that the letter J and an odd number will turn up?
   	      	           
   	      	           \begin{columns}[b]
   	      	           	\column{0.5\textwidth}{
   	      	           		\begin{center}
   	      	           			\includegraphics[width=0.5\linewidth]{Images/dice}
   	      	           		\end{center}
   	      	           	}
   	      	           	\column{0.5\textwidth}{
   	      	           		\begin{center}
   	      	           			\begin{tikzpicture}[scale=0.5]
   	      	           			\pie [sum=5, text=inside,
   	      	           			before number=\phantom,after number=]
   	      	           			{1/J, 1/K, 1/L, 1/M,1/N}
   	      	           			\node [scale=1.4, rotate=90](note) at (0,0) {\Huge $\twoheadrightarrow$};
   	      	           			\end{tikzpicture}
   	      	           		\end{center}
   	      	           	}
   	      	           \end{columns}
   	      	        \end{frame}
   	      	        
   	      	           \begin{frame}
   	      	        	\frametitle{Question of the day - Probability}
   	      	        	A 6 number die is tossed and the spinner spun at the same time. What is the probability that the letter J and an odd number will turn up?
   	      	        	\\
   	      	        	\bigskip
   	      	        	\pause
   	      	        	First we need to determine the sample space.
   	      	        	\\
   	      	        	\bigskip
   	      	        	\pause
   	      	        	$	\begin{array}{ccccccc}
   	      	        	& \bluebf{J} & \bluebf{K}  & \bluebf{L}  &  \bluebf{M} &  \bluebf{N} \\ 
   	      	        	\redbf{\Cube{1}}     	& J1 & K1 & L1 & M1 & N1  \\ 
   	      	        	\redbf{\Cube{2}}       	& J2 & K2 & L2 & M2 & N2  \\ 
   	      	        	\redbf{\Cube{3}}       	& J3 & K3 & L3 & M3 & N3  \\ 
   	      	        	\redbf{\Cube{4}}      	& J4 & K4 & L4 & M4 & N4  \\ 
   	      	        	\redbf{\Cube{5}}     	& J5 & K5 & L5 & M5 & N5  \\ 
   	      	        	\redbf{\Cube{6}}   		& J6 & K6 & L6 & M6 & N6 
   	      	        	\end{array} $
   	      	        	
   	      	        \end{frame}
   	      	        
   	      	        	\begin{frame}
   	      	        		\frametitle{Question of the day - Probability}
   	      	        		A 6 number die is tossed and the spinner spun at the same time. What is the probability that the letter J and an odd number will turn up?
   	      	        		\\
   	      	        		\bigskip
   	      	        		What are the favorable outcomes?
   	      	        		\\
   	      	        		\bigskip
   	      	        		$	\begin{array}{ccccccc}
   	      	        		& \bluebf{J} & \bluebf{K}  & \bluebf{L}  &  \bluebf{M} &  \bluebf{N} \\ 
   	      	        		\redbf{\Cube{1}}     	& J1 & K1 & L1 & M1 & N1  \\ 
   	      	        		\redbf{\Cube{2}}       	& J2 & K2 & L2 & M2 & N2  \\ 
   	      	        		\redbf{\Cube{3}}       	& J3 & K3 & L3 & M3 & N3  \\ 
   	      	        		\redbf{\Cube{4}}      	& J4 & K4 & L4 & M4 & N4  \\ 
   	      	        		\redbf{\Cube{5}}     	& J5 & K5 & L5 & M5 & N5  \\ 
   	      	        		\redbf{\Cube{6}}   		& J6 & K6 & L6 & M6 & N6 
   	      	        		\end{array} $
   	      	        		
   	      	        	\end{frame}
   	      	        	
   	      	        	\begin{frame}
   	      	        		\frametitle{Question of the day - Probability}
   	      	        		A 6 number die is tossed and the spinner spun at the same time. What is the probability that the letter J and an odd number will turn up?
   	      	        		\\
   	      	        		\bigskip
   	      	        		What are the favorable outcomes?
   	      	        		\\
   	      	        		\bigskip
   	      	        		$	\begin{array}{ccccccc}
   	      	        		& \bluebf{J} & \bluebf{K}  & \bluebf{L}  &  \bluebf{M} &  \bluebf{N} \\ 
   	      	        		\redbf{\Cube{1}}     	& \greenbf{J1} & K1 & L1 & M1 & N1  \\ 
   	      	        		\redbf{\Cube{2}}       	& J2 & K2 & L2 & M2 & N2  \\ 
   	      	        		\redbf{\Cube{3}}       	& \greenbf{J3} & K3 & L3 & M3 & N3  \\ 
   	      	        		\redbf{\Cube{4}}      	& J4 & K4 & L4 & M4 & N4  \\ 
   	      	        		\redbf{\Cube{5}}     	& \greenbf{J5} & K5 & L5 & M5 & N5  \\ 
   	      	        		\redbf{\Cube{6}}   		& J6 & K6 & L6 & M6 & N6 
   	      	        		\end{array} $
   	      	        		
   	      	        	\end{frame}
   	      	        	
   	      	        	\begin{frame}
   	      	        		\frametitle{Question of the day - Probability}
   	      	        		A 6 number die is tossed and the spinner spun at the same time. What is the probability that the letter J and an odd number will turn up?
   	      	        		\\
   	      	        		\bigskip
   	      	        		\fbox{$Probability = \dfrac{3}{36} \Longrightarrow \dfrac{1}{12} $}
   	      	        		\\
   	      	        		\bigskip
   	      	        		$	\begin{array}{ccccccc}
   	      	        		& \bluebf{J} & \bluebf{K}  & \bluebf{L}  &  \bluebf{M} &  \bluebf{N} \\ 
   	      	        		\redbf{\Cube{1}}     	& \greenbf{J1} & K1 & L1 & M1 & N1  \\ 
   	      	        		\redbf{\Cube{2}}       	& J2 & K2 & L2 & M2 & N2  \\ 
   	      	        		\redbf{\Cube{3}}       	& \greenbf{J3} & K3 & L3 & M3 & N3  \\ 
   	      	        		\redbf{\Cube{4}}      	& J4 & K4 & L4 & M4 & N4  \\ 
   	      	        		\redbf{\Cube{5}}     	& \greenbf{J5} & K5 & L5 & M5 & N5  \\ 
   	      	        		\redbf{\Cube{6}}   		& J6 & K6 & L6 & M6 & N6 
   	      	        		\end{array} $
   	      	        		
   	      	        	\end{frame}
   	     \subsection[]{Fancy Pi}
   	     \begin{frame}
   	     	\frametitle{Fancy Pi}
   	     	
   	     	   \def\defaultstartht{20pt}
   	     	   \diminish[0.97]{\pinum}
   	     	   
   	     \end{frame}
   	     

\end{document}


